%\documentclass[handout]{beamer}\mode<handout>{\usetheme{default}}
%
\documentclass[presentation]{beamer}\mode<presentation>{\usetheme{AMSBolognaFC}}
%\documentclass[handout]{beamer}\mode<handout>{\usetheme{AMSBolognaFC}}
%%%%%%%%%%%%%%%%%%%%%%%%%%%%%%%%%%%%%%%%%%%%%%%%%%%%%%%%%%%%%%%%%%%%%%%%%%%%%%%%
\setbeamertemplate{bibliography item}{\insertbiblabel}
\usepackage{plp-aixia-2021-talk}
%%%%%%%%%%%%%%%%%%%%%%%%%%%%%%%%%%%%%%%%%%%%%%%%%%%%%%%%%%%%%%%%%%%%%%%%%%%%%%%%
\title[PLP in \twopkt{}]{Probabilistic Logic Programming in \twopkt{}}
%
\author[Dellaluce \and Calegari \and \sspeaker{Ciatto}]{
    Jason Dellaluce$^*$ \and Roberta Calegari$^\dagger$ \and \speaker{Giovanni Ciatto}$^*$
    \\\smallskip\small
    \email{jason.dellaluce@studio.unibo.it} \and \email{roberta.calegari@unibo.it} \and \speaker{\email{giovanni.ciatto@unibo.it}}
}
%
\institute[UniBO]{
    $^*$ \disi
    \\
    $^\dagger$ \almaai
    \\
    $^{*\dagger}$ \unibo
}
%
\date[AIxIA, Dec. 1, 2021]{
    $20^{th}$ International Conference of the
    \\
    Italian Association for Artificial Intelligence (AIxIA)
    \\\medskip
    Milan, 2021-12-01 (Virtual Conference)
}
%
%%%%%%%%%%%%%%%%%%%%%%%%%%%%%%%%%%%%%%%%%%%%%%%%%%%%%%%%%%%%%%%%%%%%%%%%%%%%%%%%
\begin{document}
%%%%%%%%%%%%%%%%%%%%%%%%%%%%%%%%%%%%%%%%%%%%%%%%%%%%%%%%%%%%%%%%%%%%%%%%%%%%%%%%

%/////////
\frame{\titlepage}
%/////////

%===============================================================================
\section*{Outline}
%===============================================================================

%/////////
\frame[c]{\tableofcontents[hideallsubsections]}
%/////////

%===============================================================================
\section{Motivation and Context}
%===============================================================================

%/////////
\begin{frame}[c]{Context}
    \begin{block}{\textbf{Probabilistic logic programming} (PLP)\ccite{riguzzi2018}}
        \begin{itemize}
            \item programming paradigm rooted in computational logic
            \item \alert{probabilistic} theories as programs
            \item logic \alert{inference} used to
            %
            \begin{enumerate}
                \item compute \alert{probabilities} for logic queries
                \item induce probabilities from data
            \end{enumerate}
        \end{itemize}
    \end{block}
    %
    \begin{block}{In this paper}
        \begin{itemize}
            \item we focus on the \alert{resolution} of probabilistic queries
            \item from a \alert{technological} perspective
        \end{itemize}
    \end{block}
\end{frame}
%/////////

%/////////
\begin{frame}[c]{Motivations}

    \begin{enumerate}
        \item Current technologies for (P)LP are \alert{technological \emph{silos}}
        %
        \begin{itemize}
            \item poor \alert{inter-platform} interoperability
            \item (P)LP facilities cannot simply be exploited \alert{``as a library''}
        \end{itemize}

        \bigskip

        \item Hard to exploit (P)LP in mainstream programming

        \bigskip

        \item Willing to extend the \twopkt{} ecosystem\ccite{2pkt-swx16} towards PLP
        %
        \begin{itemize}
            \item our inherently multi-platform LP ecosystem
        \end{itemize}

        \bigskip

        \item Call to arms for anyone interested

    \end{enumerate}

\end{frame}
%/////////

%/////////
\begin{frame}[c]{Background -- PLP}

    \begin{itemize}
        \item Two major technologies for PLP: $\begin{cases}
            \text{\problog\ccite{de-raedt-2007}}
            \\
            \text{\cplint\ccite{riguzzi-2007}}
        \end{cases}$

        \bigskip

        \item Both based on
        %
        \begin{itemize}
            \item logic programs with annotated disjunctions\ccite{de-raedt-2007} (\alert{LPAD})

            \item Sato's \alert{distribution semantics}\ccite{sato1995}
        \end{itemize}

        \bigskip

        \item Both relying on \alert{knowledge compilation}\ccite{darwiche2002knowledge} to reach efficiency

        \bigskip

        \item Both exploiting some \alert{Prolog technology} behind the scenes
        %
        \begin{itemize}
            \item \problog{} $\longrightarrow$ YAP Prolog\ccite{CostaRD12}
            \item \cplint{} $\longrightarrow$ SWI-Prolog\ccite{WielemakerSTL12}
        \end{itemize}

    \end{itemize}

\end{frame}
%/////////


%===============================================================================
\section*{}
%===============================================================================

%/////////
\frame{\titlepage}
%/////////

%===============================================================================
\section*{\refname}
%===============================================================================

%%%%
\setbeamertemplate{page number in head/foot}{}
%/////////
\begin{frame}[c,noframenumbering]{\refname}
%\begin{frame}[t,allowframebreaks,noframenumbering]{\refname}
%	\tiny
	\scriptsize
%	\footnotesize
	\bibliographystyle{plain}
%	\bibliographystyle{alpha}
	\bibliography{plp-aixia-2021-talk}
\end{frame}
%/////////

%%%%%%%%%%%%%%%%%%%%%%%%%%%%%%%%%%%%%%%%%%%%%%%%%%%%%%%%%%%%%%%%%%%%%%%%%%%%%%%%
\end{document}
%%%%%%%%%%%%%%%%%%%%%%%%%%%%%%%%%%%%%%%%%%%%%%%%%%%%%%%%%%%%%%%%%%%%%%%%%%%%%%%%
